\documentclass{article}

% Replace `letterpaper' with `a4paper' for UK/EU standard size
\usepackage[a4paper,top=2cm,bottom=2cm,left=2.5cm,right=2.5cm,marginparwidth=1.75cm]{geometry}

\usepackage{amsmath}
\usepackage{graphicx}
\usepackage[colorlinks=true, allcolors=blue]{hyperref}
\usepackage{multicol}
\usepackage{float}
\usepackage[numbib,nottoc]{tocbibind}


\title{\textbf{\huge{A practical guide for graduation projects in the faculty of Mechanical Engineering in Delft University of Technology}}\\Version: 1.1 -- 11/2022}

\author{%
\begin{tabular}{c} \Large{Carey L. Walters}\footnote{Mekelweg 2, 2628 CD Delft, Building 34B – 1.310 $\vert$ phone: 015 27 83 705 $\vert$ mail: c.l.walters@tudelft.nl} \\ Ship and Offshore Structures, \\ Maritime and Transport Technology 
 \end{tabular} \and
\begin{tabular}
{c} \Large{Othon A. Moultos}\footnote{Leeghwaterstraat 39, 2628 CB Delft, Building 34K – 0.250 $\vert$ phone: 015 27 81 307 $\vert$ mail: o.moultos@tudelft.nl $\vert$ web: https://omoultosethtudelft.github.io/web/
} \\ Engineering Thermodynamics, \\ Process \& Energy
 \end{tabular} \and
 }
\date{}


\begin{document}

\maketitle 
\thispagestyle{empty}

%\begin{multicols}{2}
\newpage
\pagenumbering{roman}

\noindent Copyright \copyright {} 2022 Carey L. Walters and Othon A. Moultos. This work is licensed under CC BY-NC-SA 4.0. To view a copy of this license, visit http://creativecommons.org/licenses/by-nc-sa/4.0/

\newpage
\topskip0pt
\vspace*{\fill}
\hspace*{\fill}{\large''\emph{Common sense is not so common.'' }
\\
\\
\hspace*{\fill} - Voltaire}
\\
\\\\
\\\\
\\\\
\hspace*{\fill}{\large''\emph{Science can amuse and fascinate us all, but it is engineering that changes the world.'' }
\\
\\
\hspace*{\fill} - Isaac Asimov}
\\
\\\\
\\\\
\\\\
\hspace*{\fill}{\large''\emph{Without vision you don't see, and without practicality the bills don't get paid.'' }
\\
\\
\hspace*{\fill} - Paul Engle}
\vspace*{\fill}


\newpage
\tableofcontents

\newpage
\pagenumbering{arabic}

\section{Preface}
Our hope is that this short guide will help the MSc students in our faculty to improve their theses and relieve some of the stress associated with starting a new research project. With this document, we do not intend, by any means, to substitute the official guidelines of TU Delft concerning graduation projects. The student should always have a thorough look at the material and information provided by the official sources and the MSc coordinators of each department. Maritime Technique (MT) has produced a document that has a lot of procedural detail and some tips for your thesis\footnote{It can be obtained from Peter de Vos (P.deVos@tudelft.nl).}. The Study Association of the master Energy, Flow \& Process Technology at TU Delft (Disputt Prescott Joule) also has a guide with useful information. 

This is a living document. Our intention is to continue updating it based on: (a) common weaknesses that we observe in theses, (b) significant changes in the standard procedures, and (c) useful feedback from students and colleagues. For the latter, please email suggestions to the authors of this document. Intentionally, this guide points to webpages or official documents as little as possible. Such sources are subject to change, and we would like this guide to be still useful even if/when such changes occur. As this document is mainly intended to be read by MSc students in 3mE, our tips will be addressed directly to them. 

\section{Choosing a graduation project and supervisor}
You should opt for a topic/field that nicely suits your scientific interests and ambitions, keeps you motivated, and provides enough challenges for you to improve your knowledge and technical proficiency. A balanced combination of the above is going to help you develop strong expertise and shape a robust engineering profile. Thus, it will most probably increase employment opportunities (note that this is subject to many variables and cannot be taken for granted). A good start is to have a careful look at the faculty’s curriculum and spot elective courses that seem interesting to you. Do not be afraid to pay a visit to professors in your department (or even in other departments) and ask them to provide more details about their courses and possible ideas for research subjects. Since you are going to work on your MSc thesis for quite a long time, choose a supervisor that you like and who has a working style that fits yours. A supervisor that has many students is popular because of something (e.g., interesting topics, reputation for being a good supervisor), but keep in mind that a supervisor with many students also has less time. You can ask in advance if the daily supervision will be outsourced to a PhD student or a postdoc. In this way, you will know who your daily supervisor will be. Another way of identifying a potential supervisor is by choosing based on a topic/method that you know you are strong at.

Experimentation, analytics (e.g., (hand) calculations), and simulations (e.g., finite elements, molecular simulation, computational fluid dynamics, equations of state) are the three most common approaches to answering an unknown problem. It would be ideal to have all three in an MSc thesis, but this is typically not possible within the time constraints. In general, at least two are necessary to gain confidence in the answer, even if they are not pursued to the same level of depth. For example, it is not uncommon to have a thesis that is mostly experimental but does small amounts of numerical modeling by a commercial software and basic analytical solutions primarily from coursework to check the answer. A very common approach is also to write a thesis in which a new analytical solution is written for an engineering problem, and it is validated based on published experimental data and simulations by commercial numerical modeling tools. Another example is a simulation-based thesis that might feature a new numerical method or advanced subroutines incorporated into someone else’s code (e.g., a commercial FEA package or an open-source academic code), but it is validated with published experimental data and calculations from textbooks or previously published analytical solutions. The balance is up to you and your supervisor, but you should be aware of this balance and make sure to use an independent method to verify your main findings. 

Our advice is that, no matter which topic/supervisor you choose, it is better focus on dealing with a couple of points/methods/approaches really well instead of trying to work with a lot of things and touch them only superficially. You can always use your strengths as a guide; Although your MSc graduation project is a good opportunity for learning new things and improving your abilities, focusing on further strengthening your strong points is a better path than trying to improve all your weaknesses. Another point that can benefit your work as an MSc student is being scientifically extroverted. Try to collaborate with other students/projects/supervisors/departments if this helps your project, but do not forget to discuss this with your supervisor though. Some supervisors organize group meetings where all students of the team gather and discuss their progress and problems. Such meetings are a great opportunity to meet other people and be exposed to different research topics and approaches. 


\section{Literature review}
\subsection{The main idea}
The first stage of your graduation project is the literature review. This is an independent and official part of your MSc, which has specific learning objectives and is graded based on the rubric provided by your department. This means that aside from the qualitative aspects (discussed below), you should be successful in passing the literature review assessment to continue with your thesis. Literature review has the following key purposes - to help you:
\begin{itemize}
\item Check that what you want to do has not been already done. Remember that in your graduation project you should not “re-invent the wheel” but work on a, as much as possible, novel idea. This may also be achieved by building on a previous project.
\item Get a grasp of the current state-of-the-art of the field/topic that you are going to work on.
\item Identify previously published knowledge that will be helpful for you. This does not refer only to methods/technicalities but also published work that can help you deepen your understanding of a subject.
\item Identify the major challenges in the field/topic that you are going to work on. This point is maybe the most important and for sure the most challenging. 
\end{itemize}

Your thesis supervisor will give you a topic/research question (or define a direction towards a topic) during the first meetings with him/her. It is alright if you do not understand the question in its entirety in the first meeting. Read up on basic references and then return with questions. It is expected that your work will be unique. This is one of the key aims of the literature study: you should review what other people have done so that you know that what you are doing has not already been done. It is also further expected that you will build on what is done before you. This is one of the secondary aims of the literature study: to gain insight into what others have done so that you can incorporate it into your own research and go further with less effort. Therefore, the best way to approach your literature study from the beginning is to assume that your research question has already been answered, then look for the answer in the literature. If you find the answer, then you need to either make your research question more ambitious, specific, or to find a different one entirely. As you go through this process, you will almost certainly find that the literature makes good progress toward answering a question that you have in your mind, but it misses it for one reason or another. That missing knowledge is your research gap. This is where you should focus. The literature study (which may very well be the first one or two chapters of your thesis) then becomes the practice of writing down the literature gap that you identified, the background that the reader needs to know to understand it, and what supporting knowledge others have gained that will help you to fill this research gap. 

Use the time that you invest in the literature review to simultaneously understand the question better and to sharpen it into a specific question that you can answer through the course of your thesis and that you find challenging and interesting. Through the course of the literature review, you will sharpen your intuition and understanding for the topic. This will enable you to clearly state your research questions for your project. It is also important to read the grading rubric for both the literature review and the thesis itself. At the end of your thesis, this is really how you will be graded. Before submitting your literature review or your thesis, try to grade yourself objectively using the rubric itself. This will help you to think critically about where your deficiencies are and possibly resolve them before submitting your work. 
To achieve the goals above, your literature review should clearly answer the following questions:
\begin{itemize}
\item What have others done for this problem/application? This can also be taken from other application areas that touch upon your subject.
\item To what degree are the published data/methods relevant to your project, and what can be learned from them?
\item What is the state-of-the-art in the topic? Remember that even if your thesis is fully theoretical or solely focused on modeling, it is necessary that you review the experimental work too, since it contributes to the state-of-the-art and allows you to validate your findings.
\item What are the problems and emerging issues in this area? This covers both the scientific and the engineering point of view. What do we take up in this project?
\item How do we approach this problem based on the state-of-the-art and the existing studies? Are the required tools available? 
\end{itemize}
When documenting your literature review, be sure to summarize it into a story line. A common mistake that students do with their first literature review is to list all the references that they found interesting and give a summary of each of these references. Not only does this result in an unnecessarily long literature review, but it is also redundant. Instead, see if you can find themes, write about those themes, and back it up with references. A good approach would be to write something that reads like that:
\\\textit{“There appears to be consensus in the literature that point A is valid} [ref] [ref] [ref]\textit{, though it is implemented in method B by some authors} [ref] [ref] \textit{and by method C by others} [ref] [ref]\textit{”.}
\\\\A bad approach would be: 
\\\textit{Author 1 makes point A and implements it by method B} [ref]\textit{. Author 2 makes point A and implements it by method C} [ref]\textit{. Author 3 makes point A and implements it by method B} [ref].\newline

It is expected that you will receive some key references from your supervisor to start with. However, it is also expected that you will dig into literature to find most of the information yourselves. Primary sources of information should be academic databases, e.g., Scopus, Web of Knowledge, and Google Scholar. You may notice that these databases give the number of times that a particular work is cited. This number tells you how many times other researchers have referenced this work and can sometimes give you a tip that a work is particularly influential. There are many ways of proceeding with a literature review, but a key way is to start with an influential work that either you or your supervisor has identified, then look through its references to see what it has cited. Most databases also show you which works have cited a particular paper, so it may be interesting to read through the papers that have cited the paper that you find interesting. This is also a way of checking the validity/importance of a paper. For example, suppose you find a key paper from some time ago in which an essential part of the research (question) is explained, or in which a particular method is introduced (that you would like to use in your research). Then it is logical to expect that most of later studies that are dealing with the same topic/method should cite this paper. In this way you can very quickly analyze what is done in the field\footnote{It is possible to see the list of papers that cite a certain paper by using Web of Science
Most often this list is also provided at the webpage of the published paper.} 
If nobody or just a few have cited this paper, it can mean various things: it is simply overlooked, other researchers do not find this interesting for a certain reason (and try to find out what this reason is, or ask your supervisor), there is a mistake in the paper (that is possibly explained in one of the papers that cite the original work), or the method is obsolete. It is also often fruitful to use the databases to look up other papers by the authors that you find to be the most helpful.  Last but not least, you can always search the databases by keywords. Later on, in section “Writing your thesis” we provide tips on writing a high-quality Literature Review and Thesis. 

\subsection{Resources}
TU Delft has plenty of resources, though they are finite. Your research question should be designed in such a way that it is answerable with the resources available to you at the university (or your host company). This includes legal software, experimental facilities, computational power, etc. You should discuss the use of these resources with your supervisor before completely settlinfg on a research question. For example, in the case of a computational subject, it is not expected that you have an a priori deep knowledge of how computationally intensive your work is going to be. This is something that should be discussed early on to ensure that there are enough computational resources for you. In some rare cases, it is possible that the project is even computationally unfeasible. Finally, we have no tolerance for using illegal software. Not only can you get fined or sued by a software manufacturer but, in most cases, it is simpler to request access to the same software (or similar) from the university. For example, in some cases your supervisor/department can buy a license to the software that you need for your research.

\subsection{How to read a journal article}
Do not sit down and read a journal article like you would a novel – starting at the beginning and ending at the end. If you do that, you will soon feel overwhelmed, while the probability of understanding most of the context at this stage is relatively low. Instead, try to get the most important points without investing too much time in it. The best practice is to read the abstract, conclusions, and main figures, in that order. For some papers, it makes sense to look quickly through the abstract. Take some notes about how that paper relates to your research questions. In the beginning of your literature review, you may encounter hundreds of articles that seem interesting, and reading them all and in full is practically impossible. When you are doing your initial scan, a rule-of-thumb is to restrict yourself to maximum thirty minutes per paper. When doing this first scan, you should start getting an impression of the main points, the main method(s) used, the things where consensus has been reached between different researchers, and who the most influential research groups/scientists are. Once you get this sense, you should return to the most important/relevant papers that are written by the most influential authors. This is when you should spend more time studying the paper – up to a day or more. Start with the same strategy: read the abstract, introduction, conclusions, figures. Identify the most interesting/important figures and find the parts of the text where they are referenced. Read about the important figures in the relevant text. Once you have done that, read the whole paper from introduction to conclusions.

Always keep in mind that journal articles are written by humans and reviewed by humans (the reviewers are typically volunteers who receive no compensation or immediate recognition and have a very long “to-do” list other than reviewing articles). This means that journal articles are very subject to errors. Do take them as being authoritative, but do not take them as being infallible. When you read them, especially the ones that are very interesting/relevant, consider whether you believe their findings and outcomes or not. Do their findings match with basic principles of science/engineering? Were the methods sound, and did their conclusions follow naturally from the evidence that they gave? Do their findings match prior, relevant studies? Are they in a peer reviewed source or not? (Note that most journals are peer-reviewed while most conferences are not). Are they in “good” journals and publishers? You may get an intuition about this by looking at the journal impact factor or its quartile, by asking your supervisor for advice, and by looking at the authors/institutions involved. For example, it is very likely that research groups from prestigious universities in Europe and USA known for excellence in research will publish their studies in highly respected journals. You should be cautious; you will quickly learn that open-access “predatory journals”\cite{predatory} (i.e., journals that require publication fees but perform no quality checks)\footnote{This type of journals should not be confused with the currently-supported practice of publishing open-access. TU Delft Library has bilateral and national agreements with many publishers ensuring that you, as a researcher, do not incur publication costs. This means that you not only have access to the vast majority of published work but also that if your work is published in a published with which TU Delft has an agreement (most major publishers) it will be open-access. More information on this can be found in TU Delft’s Policy on Open-Access Publishing and in TU Delft Strategic Plan Open Science 2020-2024.}\footnote{A list of predatory journals: \url{https://beallslist.net.}}, are becoming increasingly prolific. Nevertheless, you cannot a priori disregard all open-access journals. You always need to have a critical assessment of the information available. It is also worth mentioning that criticality is a strong quality for an engineer (and an important element on the thesis grading rubric), and this type of analysis is how you demonstrate your criticality. Do not hesitate to ask your supervisor’s opinion on the quality of a journal. He/she has enough experience to help you avoid losing your time with low quality publishers/journals. 

\subsection{Formulating your research question}
As we discussed earlier, your supervisor will provide you with an initial broad research question or a clear direction towards a topic (if none of the two happen, and given that you do not have a very strong own idea yourself, then you should probably consider another supervisor). You should always be the one to make the final formulation of the research question. This is not because your supervisor is unable or unwilling to do that but because scientific research is “fluid”. This means that if the topic you are going to work on is novel then advances are going to happen every day, and therefore, many research questions are continuously being answered. This is also the reason that your supervisor will often prefer to keep the initial topic broad enough. In a way, the literature review serves as a filter to move from the “big”, high-level question that is given at the beginning of the thesis to the specific, solvable problem that is found at the end of the literature review. Most thesis supervisors do not expect for you to solve the “big” question that you get at the beginning of the thesis. This is possibly a question to mention in your introduction as motivation for your research, but it is likely intractable to solve in a single MSc thesis. At the end of the literature review, you will know what is already published in the field, what is possible, and what you want for your contribution to be. For example, you might move from a question like:
\\\textit{What is the effect of having a high ratio of yield stress to ultimate tensile stress on the safety of a structure?}
\\\\to a question like:
\\\textit{How does a high ratio of yield stress to ultimate tensile stress in steels affect the magnitude and extent of plastic yielding in the vicinity of a circular stress concentration?}
\\\\The first question leaves a lot of open ends, such as:
\begin{itemize}
    \item What is meant by safety? Is this a reliability question or safety against a particular failure mechanism?
    \item If a particular failure mechanism is inferred, then which one?
    \item What kind of structure? All of them?
    \item What material(s)?
\end{itemize}
It is highly unlikely to answer the first question in its entirety in a MSc thesis, but the second one could be solved. Furthermore, at the end of a literature review, the student should know what is already available for the second question, such as an existing elastic solution, some theory on how to incorporate plasticity, existing test data, etc.

The gaps identified in your literature review should give you a strong idea about what the main theme of your research should be. However, your research question should be further refined to become part of your thesis. When posing your research question, consider:
\begin{itemize}
    \item It should not be a yes or no question.
    \item	How will you know if you have successfully answered the question?
    \item	Has it been answered before, including works in your literature study?
    \item	Is it solvable with the resources at your disposal within the time of your graduation project?
\end{itemize}

Your research question should have a reasonable balance between being broad and specific. If it is very specific to a process/product, then you need to think hard about whether you are filling an important scientific knowledge gap or just working as an engineer for a company. Nevertheless, the problem must be narrow enough that it can be solved within the time of your graduation project. Keep in mind that you can always adjust/re-shape your research question during your graduation project. Remember that it is better to partially solve an important problem, create a robust foundation, and a clear future outline, than trying to fully solve a problem that is not very relevant or important to your field. 

\section{Managing your graduation project in an efficient way}
\subsection{Update meetings}
Although you are dealing with one supervisor, your supervisor is most probably dealing with multiple students. Therefore, you should take the initiative in organizing update meetings. The frequency and duration of these meetings is something that you can discuss early on in your project. It is your right to expect serious and professional supervision, feedback, and guidance. Thus, it is your right to spend enough time with your supervisor to ensure your progress. Whether this time is one or more hours a week is something to be agreed upon. In any case, the time that you meet with your thesis supervisor is precious. Therefore, it is of utmost importance that you come prepared to make the most out of this meeting. For example, it is important to have notes on what you want to discuss and a presentation or clear notes with your latest results. Relevant research papers should be either printed or easy to find on your computer. An efficient way to start the meeting with your supervisor is by briefly listing the conclusions/discussion points from the previous meeting. This will help your supervisor to immediately focus on your project. Recall that your supervisor has many different projects running and he/she cannot always remember things from previous meetings, so helping him/her a bit can make quite a difference.You should also set the agenda of the meeting, even if the agenda itself is reasonably flexible. It is recommended that the agenda include at least:
\begin{itemize}
    \item The status in the project relative to goals and the end date.
    \item A short summary of the action points from the previous meeting.
    \item	Technical progress in your thesis.
    \item	Questions for your supervisor.
    \item	Requests for resources (e.g., access to computers, lab space, advice, or software). 
    \item	The date and time of the next meeting.
\end{itemize}
By the end of your meeting with your supervisor, the following points should be added to the list above:
\begin{itemize}
    \item	A to-do list for the coming days until your next meeting.
    \item	A list of actions for both you (and your supervisor).
\end{itemize}

Many of the academic staff at the university have very busy schedules, and they are very likely to have meetings scheduled after yours. Therefore, make sure you are not late for the meeting.  It is also highly encouraged that you arrange the agenda of the meeting such that the most critical parts are in the beginning of the meeting, and less important parts are at the end, where they are more likely to be cut off by the next meeting in case the discussion on the important parts takes more time than initially scheduled. 

\subsection{Planning}
TU Delft graduation projects should last a total of nine months. This time includes the initial literature review phase and the thesis writing. Therefore, the student should realistically plan for at least 6-8 months of pure research. The final draft of your thesis should be ready at least 2 weeks (10 working days) prior to your defense so the committee members have enough time to read it and prepare. This means that the final draft thesis you for your supervisor(s) should be handed in earlier. This “internal” deadline is something you can discuss with him/her; however, keep in mind that your supervisor will need time to go through your thesis and provide feedback. If you do not provide your draft in time, your supervisor will have limited time to assess it and provide useful feedback. This will lead to a report of lower quality (and possibly of lower mark). Thus, it is important that you develop a plan by the end of your literature review that will help you to meet this time schedule. The plan is often communicated by a Gantt chart, though a “critical path” is also good. Either way, dependencies between the various actions and their durations should be indicated in a clear way and adhered to. Risk should also be considered into this, such that the various actions might be indicated as a range of minimum and maximum time to execute. A clever plan is also to already start devoting time to familiarize yourself with the tools (e.g., numerical/simulation methods and programs/codes, experimental apparatus, etc.) that are you going to use during your thesis while performing your literature review. Such an approach has multiple benefits, e.g., you save time by getting hands-on experience early on, you avoid the monotony from only studying (and planning) during your literature review phase, and you encounter possible practical/technical limitations and difficulties that can help you better shape your research question(s). 

\subsection{What if things do not go well}
As with every aspect of life, MSc graduation projects do not always go as planned. This can be for many reasons e.g., the topic was proved to be too difficult or even not possible to solve within a reasonable timeframe, the chemistry/relation with the supervisor is not the best, personal/health issues. All these reasons (and many more) can significantly hinder progress in your project and, consequently, cause delays in or even jeopardize your very graduation. In such cases, remember that you are not alone. You can always ask for help and guidance. If things are not going well, talk to your supervisor, or if you feel uncomfortable, talk to a student counselor, a colleague of your supervisor, or fellow MSc students. It is important that you do not postpone seeking help if you feel unhappy with the project or with your supervisor. In case you have a serious conflict with your supervisor or if you feel you are a victim of any kind of inappropriate/unprofessional/unwanted behavior by anybody involved in your project, it is time to act. Talk to a Confidential Advisor (vertrouwenspersoon)\footnote{The Confidential Advisor has a duty of confidentiality, and thus, you can expect none of your complaints to be openly discussed.} of TU Delft as soon as possible.  Rest assured that if your problem purely revolves around the science of your project, you can most probably find a solution along with your supervisor. If you are dealing with a personal/health issue, communicate it as soon as possible so that the department can provide the best possible help. Remember, your health (both physical and mental) is far more important than your graduation project or any form of study. 

\subsection{Data management}
During your graduation project you will produce data. Part of these data will comprise your results in the report/presentation, while it is possible, that a large portion of these data will be less useful or even contain errors. It is also possible, that you will be either handed in datasets from an already running project that you can build on or that you will be requested to hand in data yourself to a fellow researcher to continue the research project from where you stopped. For the latter to happen in an efficient way, the datasets that you pass to your colleague(s) will need to be managed and stored properly. This means that it is imperative to often scrutinize your datasets, keep track of data that contain errors, and store everything in a meaningful and, as much as possible, easy-to-understand manner. For example, if during your project you measured the shear viscosity of a liquid substance at different system temperatures and pressures, you will need to store your results in a way that it is clear to a fellow researcher which value of the shear viscosity corresponds to every temperature and pressure. This example is rather simple, however, such datasets may span hundreds or thousands of lines, and include multiple inputs/conditions/variables making their processing a cumbersome task. A proper way of storing such datasets would be to create properly named individual text files or spreadsheets in properly named folders. Your storing method needs to include all measured/computed quantities with units, the date when you performed your measurement/calculation, clear notes explaining how you ended up to a specific number (if needed), and any other piece of information that is necessary for another researcher to be able to process and reproduce your data. Proper data management becomes even more important when you are dealing with confidential or sensitive data, for example datasets including confidential information of people or industrial practices/processes. For more information on this topic, you can refer to the Research Data Management policy of TU Delft and/or talk to a Data Steward.

\subsection{Safety}
Laboratory work can introduce safety risks, which can even include the use of materials and situations that may appear familiar and safe outside of a laboratory setting. TU Delft takes safety very seriously. Therefore, when laboratory work is necessary for the execution of a thesis, the student must take responsibility for the safety of the work. This takes the form of preparing an experimental plan that is agreed upon with the daily supervisor and the “area supervisor” responsible for the relevant laboratory in advance. The experimental plan must include a realistic identification of risks associated with the laboratory work and what methods are used to mitigate these risks. The area supervisor will inform you of the lab rules, assess the safety report, assess your capacity to work safely in the lab (e.g. by a test on what to do in case of an emergency), and upload your plan into the university’s laboratory safety management system (LabServant). Your daily supervisor should be the first point of contact between you and the area supervisor, but if you have doubts, then reach out to your department manager or the Safety Advisor for 3ME (Peter Kohne).  Either of these people should know who the “area supervisor” is for the relevant lab and know how to interface with TU Delft’s safety system.

\section{Writing your thesis and literature review} 
\subsection{A general guide}
Here we will provide with some guidance and practical tips for preparing a high-quality report. These tips apply to both your literature review and your final thesis. TU Delft has a template for MSc theses (both in latex and word). Please use this as a starting point. Writing a high-quality thesis takes time. For this reason, you should plan ahead and start writing in e.g., the 7$^{\text{th}}$ month of your project or earlier. A clever approach to this would be to already start drafting sections that are relatively easy to write as early as possible if you have time while waiting for simulations or experiments to finish. Such sections can be the methodology, details on the experiments and/or modeling, and materials/codes used. Your supervisor will have comments and feedback on the first draft(s) of your thesis. Incorporating this feedback also takes time, and you should consider it in your initial planning. We recommend organizing the writing of your thesis as follows:
\begin{enumerate}
    \item You write a working title that you use for paperwork and to communicate with your supervisor.
    \item Then you write a working abstract. Think of this as an elevator pitch of half a page that you give to peer / non-expert in the field. Be clear and succinct. More details on writing the abstract are given below. 
    \item Then write the table of contents. This is a way of giving your supervisor a general outline of what you think the main parts of your thesis will be. This can change with time, but it is important to establish an idea of the general outline early.
    \item Then you write the introduction (and research question). This should be the direct result of the literature study phase. The introduction should clearly place your work in the broader literature and show the research gap. Another important aspect to cover in the introduction is to clearly articulate the impact of your study: Who will benefit from this research and what is the impact of your work on the society/technology/science etc. 
    \item After performing the technical part of your thesis, you write your conclusions. These conclusions should answer your research question as concretely and completely as possible. They should be as clear and succinct as possible and focus exclusively on the findings of your own research. They should not include general observations or background information but clearly show what did you learn from your study and what is the impact of your findings. Depending on the conclusions (i.e., practically your findings), you may consider fine-tuning your research question(s).
    \item After you write the conclusions, you write the intervening chapters between your research question and the conclusions. These chapters should only contain the necessary information for the reader to clearly understand how you arrived at your conclusions, why you believe that they are correct, and what limitations exist. If a piece of information does not help you to answer the research question, explain how you did it, show why you think that your answer is correct, or clearly show limitations, then it does not belong in your thesis. Keep in mind that your committee knows that it can be painful to leave something out that you fought hard to learn but including it does not make for a better thesis. 
    \item The last chapter of your thesis should be one to cover recommendations and a future outlook. Here, you can recommend what the next steps for extending your work should be. The way of approaching this chapter is by thinking what your actual steps would be if you had more time to work on your subject. You can cover both the short- and the longer-term actions that could potentially improve/benefit the research you have performed so far. Aside from these actions, a broader outlook for useful future research in this field is important. 
    \item Depending on your thesis, the background section may be part of the introduction, a separate chapter, or embedded in the various chapters. Any way it is introduced, the background should contain general information that you believe a well-trained non-specialist engineer needs to understand the developments in your thesis. It is recommended that you write this as you are writing the technical chapters (next bullet). If you are writing something that requires a lot of background information to understand, then either introduce it in the relevant chapter or note it in the dedicated background chapter. General background that is not required to understand your thesis should be left out.
    \item Limitations should be clearly stated in your thesis. They could be introduced in different sections, depending on your particular thesis, but they should be mentioned.
    \item Depending on how you execute your thesis and your agreement with your supervisor, you might write your methodology as you are doing the thesis, thus before the conclusions. If you do this, then be prepared to revisit it after you write the conclusions to remove parts that have become less relevant during the development of your thesis.
    \item You revisit your title and abstract to make sure that they still represent your research question and conclusions.
\end{enumerate}

\subsection{Deciding on an appropriate title}
The title of your thesis does not need to be the same as the one given on the forms. It can be (and in most case is) revised after the rest of the thesis is complete. The title should be as specific as possible so that the reader will instantly know whether or not the thesis is of interest to them. Titles that focus on the conclusions are generally better than ones that focus on methods or motivation. Also, when you generate your title, it is best to focus on the unique part of your thesis, rather than common parts that many other theses or papers have. As much as possible, avoid using qualitative terms (e.g., “high”, “low”, “a variety of”), but instead use quantitative (e.g., “temperatures between -60 °C and room temperature”). Finally, your title should be as short as possible. Remember that you have an abstract to try to further convince readers that they want to read your thesis.

\subsection{Writing the abstract}
The abstract is a very special part of the thesis (and of a scientific report/paper in general). The purpose of the abstract is to condense the entire thesis into a very short piece of text (about half of an A4) so that a casual reader can figure out if they want to read the rest of your work. By the time you write your abstract, you will certainly have encountered and used many abstracts during your literature review. Consider what you liked/did not like in an abstract. Consider what made it useful for you to use in deciding whether or not to read the rest of the paper. Essentially, the abstract is an executive summary/elevator pitch of your work. 

When you write the abstract, start with motivating your work (1-2 sentences). This means that within these sentences you should cover what the topic is, why this topic is relevant/urgent/important for science and technology, what is the novelty of your work, and which open research questions you answer. The end of your abstract should have 2-3 sentences that summarize your conclusions as quantitatively and succinctly as possible. This means that only the most important findings of your work should be written here, and not all intermediate findings or test cases you tried. Ideally, a sentence can be added here to describe the impact of your findings on science/technology/society. Between the motivation and the conclusions, you should describe in a few sentences how you arrived at your conclusions, especially focusing on your methods. Again, do not list the methods if full detail but rather describe what is the main methodology and tools you used. In case you have developed a new methodology, this part can be more elaborate. Depending on your thesis, it may be useful to indicate any important assumptions. 

The abstract is $not$ the introduction and should not repeat the introduction. It is impossible to reproduce the entire thesis in the abstract, so do not try to do that. Instead, use this as your chance to convince the right audience (those who would be interested in your work) that they should read it. The abstract should contain no new information, but only summarize what can be found elsewhere in the main text. It is especially important to use clear, short language.

\subsection{Writing the introduction}
The introduction has a special place in your thesis and must accomplish several goals at once:
\begin{itemize}
    \item Motivation:
    \begin{itemize}
        \item Societal relevance: why is this important from a societal point of view?
        \item Scientific relevance: why is this important from a scientific point of view; what is the research gap?
    \end{itemize}
    \item Problem definition and/or research question.
    \item Possibly introduce the structure of the thesis.
\end{itemize}

The introduction should be written in the form of prose (i.e., not bulleted), though the research question is often called out or highlighted in way, such as italicized or indented. More so than anywhere else in the thesis, the introduction should be self- contained – i.e., you should not need to have specialist prior knowledge or information from elsewhere in the thesis to understand it properly.

Depending on the structure and focus of your thesis, you may outline the methodology in your introduction, e.g., in case you developed a new method or heavily optimized an already existing one. Depending on the structure of your thesis, your introduction may also include background information that the reader will need to know to understand the rest of the thesis. It is important to note that background information does not mean to copy and paste large pieces of already published text or reproduce (and elaborate on) well-known scientific facts. In sharp contrast, background information should be a balanced and succinct piece of text having the purpose of helping the reader get a grasp on the research area.

\subsection{Writing the methodology section}
The methodology section serves two main and equally important purposes:
\begin{itemize}
    \item To provide sufficient information that other researchers can use to reproduce the findings of your thesis. This means that all experimental or computational methods you used should be clearly stated along with specific details on parameters and/or equipment. A reasonable (and rather obvious) way of ensuring that this purpose is served with your methodology is to read it and try to see if with the information you provided you could reproduce all your simulations/experiments. 
    \item To make clear to the reader where you stand in the technical state of the art in your field. This is important since you can always perform computations/measurements in many ways, some of which may be obsolete because newer and more advanced methods have been developed. A big part of the scientific research is devoted to updating the computational/experimental methods not only to extend their capabilities, but in many cases, to prevent inherent inaccuracies associated with them. 
\end{itemize}

A way of learning how to write a clear methodology section is by looking into the respective sections in high-quality scientific publications relevant to your work. Note that if in your thesis you have worked with well-established and/or widely used methods, there is no need for explaining all the details of these methods in detail. It is sufficient to clearly mention the method and cite the appropriate source. For example, there is no value in copying into your thesis lengthy derivations of the equations you used if these equations have been extensively used in your field. Similarly, if you have been using commercial or open-source software that is considered standard in your field of research, there is no point in giving all the details about the inner workings of these codes (unless you have modified the codes). In that case, it would be much better for the reader if you provide the input files you used as an appendix in your thesis. Again, if these input files are common or easy to create, the added value of including them into your thesis is not high. On the other hand, if during your graduation project you have developed a new method or devised a novel experimental setup, then it is imperative to describe all relevant aspects of the new developments in detail. In any case, discussing the limitations of the methods you used (e.g., the validity range, applicability to specific systems) is also an important part of this section.

\subsection{Practical tips for writing a high-quality literature review and thesis}
\begin{itemize}
    \item Do not write things in your thesis that you do not understand yourself.
    \item Write your thesis/literature review to be as short as possible, i.e., aim for a maximum of 50-60 pages for a thesis. Do not worry that you will have enough to say; the typical problem is that theses are too long. Your graduation committee will be more impressed by a succinct, clear contribution than a large page count.
    \item You should write your thesis for a well-educated, non-specialist engineer/scientist. Said another way, you should write your thesis in such a way that you would have understood the thesis if you read it when you started, and your friends from your classes will also understand it without you having to give a separate introduction. A common mistake is that the students write their thesis with their advisor in mind. This results in a thesis that is so filled with detailed knowledge and jargon that even some graduation committee members cannot fully understand.
    \item Prefer to use short sentences of simple construction. Do not use redundant words. Limit the use of words such as $hence$, $therefore$, $additionally$, $moreover$, $however$, etc. It is very common that these words are used in excess, and therefore, worsening the quality of the document. 
    \item Always use proper notation and terminology. For example, the ISO 31\footnote{\url{https://www.iso.org/committee/46202/x/catalogue/}} and IUPAC\footnote{\url{https://iupac.org/what-we-do/nomenclature/}} nomenclature for symbols (i.e., measurable quantities in italics, vectors in bold, sub- and super-scripts not in italics except if they are measurable quantities, etc.) helps to obtain a uniform and consistent notation in your document. It is strongly preferred to use metric prefixes in lieu of scientific notation, e.g., It is better to say that the specimen is 10 $\mu$m or 0.01 mm instead of 10 E-5 m or 10 x 10-5 m. 
\end{itemize}  

Purdue University has published a thorough writing guide\footnote{\url{https://owl.purdue.edu/owl/general_writing/}} that includes a lot of detailed tips on every level of writing. English for Academic Purposes 3 (EAP 3) is recommended, and it gives an opportunity to get feedback on your thesis from a purely writing perspective. “Getting it across” by Sören Johnson is recommended. Last but not least, we highly recommend the great video on “The Craft of Writing Effectively” by Larry McEnerney from University of Chicago\footnote{\url{https://www.youtube.com/watch?v=vtIzMaLkCaM}}.
    
\subsection{Practical tips for the figures}    
\begin{itemize}
    \item The figures should tell a story. Make sure that your figures convey a clear message to the reader. This means that you should include in your thesis only the figures that do so. Also, make sure that you cannot include a figure if you do not refer to it.
    \item Pay much attention in having high-quality plots. To achieve that, you should make axes labels and symbols easy to read, the axes and tick marks wide enough, and lines/symbols within a figure sufficiently large. Also, choose the symbols and colors carefully so the reader can see the difference between the different datasets/lines. Always label graph axes, including units. You may consider omitting these during informal meetings with your supervisor, but this is not a good practice. We recommend including them even early in informal settings because it will then be easier to inherit them into your formal documentation and minimize errors.
    \item Figure captions and legends are very important. Legends should be written in a font size that can be clearly read (similarly to the axes labels). The caption should be written in such a way that the figure is self-explanatory. This means that one should be able to understand the figures without reading the main text.
    \item Ensure that the notation used in the figures is the same as in the rest of the document (see previous bullet point). Ensure that plots are “printed” with high-resolution (e.g., $>$400 dpi). Equal attention should be paid also to making high-quality tables (e.g., proper alignment and notation, font sizes etc.).
    \item Tables, Equations, and Figures must be numbered in the order of appearance in the text. Never refer to an equation that is not yet presented in the text.
    \item A proper way of presenting the units of your figure axes is “$quantity$ / [units]”. 
    \item It is common to want to use data from another source, e.g., in case you want to compare your analysis to a previously published plot. It is rather annoying for the reader to compare two plots, so it is strongly preferred if the prior data is plotted into the same plot as your data. I strongly recommend using a digitizer in this case. A free on-line digitizer is the WebPlotDigitizer\footnote{\url{https://apps.automeris.io/wpd/}}. Of course, when using someone else’s data, you must reference it properly.
\end{itemize}

 


\subsection{Grammatical/textual pet peeves}
\subsubsection{How to ...}
Many TU Delft students like to propose questions in the form of “How to do that?” This is not grammatically correct. A simple alternative is “How can that be done?”

\subsubsection{Commas}
The use of commas is very straighforward because there are only a few simple rules. Please memorize and apply the following rules:
\begin{itemize}
    \item The comma does not separate a subject and a verb. E.g., “He, said that it is a bad idea” is indeed a bad idea.
    \item Do not comma splice! In other words, a comma alone cannot separate two independent clauses. E.g., “He used a comma wrong, it was terrible!” is indeed terrible. There are three easy fixes to the comma splice: switch it into a semicolon, add a conjunction after the comma, or change the comma into a period. E.g., consider “He used a comma wrong; it was terrible!”, “He used a comma wrong, and it was terrible!”, or “He used a comma wrong. It was terrible!”.
    \item A semicolon alone can separate two independent clauses. E.g., “He switched his comma splice to a semicolon; that made it all better.” is indeed better.
    \item Whenever a conjunction separates two independent clauses, the conjunction should be preceded by a comma. E.g., “He used a comma right, and it was good!” is correct with comma.
    \item Introductory phrases are separated from the rest of the sentence by a comma. E.g., “In order to get a better grade, the students all placed their commas correctly.” is indeed a correct use of a comma.
\end{itemize}

\subsubsection{Always, never, not possible}
The words “always” and “never” are very strong words. If you use these words in your thesis, you should be prepared to defend them and answer any exceptions that the committee may ask about. “Not possible” is not the same as “very difficult” or “impractical”. If you say that it is “not possible”, then you should be prepared for a committee member to ask you if it is possible if you had more time, equipment, etc.

\subsubsection{Mass nouns}
A mass noun is a noun in which it is assumed that there is an uncountable amount, so it is stated as singular. The word “research” is a mass noun. Therefore, one does not “do a research”, and one does not review “previous researches that were done”. Rather, one “does research” or reviews “previous research that was done”. Other important mass nouns include “education”, “effort”, “advice”, and “infrastructure”. “Insight” can be a count noun or a mass noun, based on context.

\subsubsection{Compound adjectives and units}
When you are using a number and its associated unit as part of a compound adjective, then the unit does not become plural. For example:
\\\\Correct: The overhead crane picked up the 4-ton pallet. (Correct: “4” and “ton” are describing “pallet”).
\\\\Correct: The pallet that the crane picked up was 4 tons. (Correct: “tons” is describing “4”).
\\\\Incorrect: The truck pulled the 2-tons trailer. (Incorrect: the “tons” is part of the descriptor of “trailer”).

\subsubsection{Do not blame your word processing software!}
Your committee members all know that word processing software (especially Word) can be annoying. However, you (and only you) are responsible for your thesis. Especially if you use software to manage your references or figure numbers, flip through to make sure that your software did not mess up your references or figure numbers before submitting a draft. Pay special attention to how standards are treated because referencing software seems to have special difficulty with them. “That reference is messed up because I used referencing software, and it did that.” is not an excuse.

\subsubsection{Construction versus structure}
In English, the word “construction” is typically heavily associated with the act of constructing. Thus, something that is planned to be built or has already been built is a structure. For example:
\\\\Good: Travel on the highway was delayed due to construction.
\\\\Good: We analyzed the offshore structure to assess its stability.
\\\\Bad: We analyzed the offshore construction to assess its stability.

\subsubsection{Nowadays...}
To American ears, the term “nowadays” sounds either very casual or very old fashioned. When Americans hear “nowadays”, they expect it to be followed with “Nowadays, those darned kids listen to their music too loud!” A better word for scientific writing is “currently”.

\subsubsection{“Used” as an adjective}
It is very unusual for native English speakers to use the word “used” as an adjective, e.g., “The used method is good.”. A good example would be: “The method that was used is good.” or (better still) “A good method was used.” Another example is to reword “The used methods in our industry…” into “The methods used in our industry…”

\subsubsection{Adverbs}
Adjectives describe nouns. Adverbs describe nouns and adjectives. Adverbs often (though not always) end in “-ly”. E.g., In “a previously published plot”, “previous” takes an “-ly” because it is describing an adjective – “published”. If it were describing just the plot (a noun), then it would be “a previous plot”.

\subsubsection{Preferred wording}
Use “equation” instead of “formula” and “figure” instead of “graph”. Try to avoid the word “not” as means of negation (see some examples in Table ~\ref{table:negation}). Try to use statements instead of long/wordy sentences (see some examples in Table ~\ref{table:wordiness}). Make proper use of collocations.
\begin{table}[H]
\centering
\caption{Examples of how to avoid the word “not” as means of negation. Source: Ref. \cite{Weiss}. }
\begin{tabular}{l l}
\hline \textbf{Example} & \textbf{Preferred usage} \\\hline
 Do not account for                 & Exclude/neglect   \\  
 Do not allow                       & Prevent           \\
 Do not have much confidence in     & Distrust          \\  
 Do not meet the requirements       & Are insufficient  \\
 Not always straight forward        & Complicated/difficult/problematic \\ 
 Not constant                       & Alternating/changing \\
 Not clear                          & Unclear/unresolved \\  
 Not economically viable            & Uneconomical/unprofitable \\
 Not explicitly discussed/not taken into account/not used & Excluded  \\  
 Not feasible/not the case          & Infeasible  \\
 Not important                      & Unimportant/negligible \\  
 Not included in                    & Excluded from \\
 Not just/only [...] but also [...] & Both [...] and [...] \\
 Not sufficiently reliable          & Unreliable \\
 Not too distant                    & Close/near \\ \hline
\end{tabular}
\label{table:negation}
\end{table}

\begin{table}[H]
\centering
\caption{Preferred usage of statements to omit wordiness and inappropriate jargon. Source: Ref. \cite{Weiss}. }
\begin{tabular}{l l}
\hline \textbf{Example} & \textbf{Preferred usage} \\\hline
 A considerable amount/number of         & Much/many   \\  
 A decreasing amount/number of           & Less/fewer  \\
 A (great/vast) majority of              & Most \\
 A small number of                       & Few \\
 As a consequence of/as a result of      & Because  \\
 As to whether                           & Whether \\
 At a rapid rate                         & Rapidly \\
 At an earlier date                      & Previously \\
 During/in the course of                 & While \\
 First of all                            & First  \\
 For the reason that/the reason why is that & Because  \\
 From the point of view                  & For \\
 In a number of cases                    & Some \\
 In a small amount of cases              & Rarely \\
 In case                                 & If \\
 In order to                             & To \\
 In the absence of                       & Without \\
 In the last/past analysis               & Previously \\
 It is apparent that                     & Apparently \\
 It is worth pointing out that           & Note that \\
 On the basis of                         & By \\
 One of the most important               & An important \\
 Regardless of the fact that             & Even though \\
 The fact that                           & Because/although \\
 The question as to whether              & Whether  \\
 This is a subject that                  & This subject \\
 Through the use of                      & By/with \\
 Would seem to indicate                  & Indicates \\
\hline
\end{tabular}
\label{table:wordiness}
\end{table}

\subsection{More on referencing}
All sentences that start with or are similar to the following should have a proper citation in them:
\begin{itemize}
    \item In the publication by…
    \item Smith reported…
    \item Smith et al.
    \item It was reported/found/shown that… (this might sometimes be supported by an internal reference, but that would indicate a wording such as “Chapter 5 of this thesis demonstrated that…”)
\end{itemize}
Any data that is not your own that you have used in your own figure (e.g., in case you compare your simulation to published experimental data) should be referenced and explicitly mentioned in the figure caption. Also, make sure that it is clear for the reader that these data are taken from another source and are not produced for this thesis. 

As much as possible, you should use primary sources. Do not cite information to a source that references another source for the information that you are giving. A corollary to this is that it should be very rare to cite textbooks. Textbooks are almost always collections of information that is first discovered elsewhere, so your preference should be to cite the source that the textbook cites. Nevertheless, if a textbook is a primary source or it states a subject in the clearest way, it should be preferred over other sources. It is never acceptable to copy text from another source (even your own) without quotes and a proper reference. It is strongly advised that you not do this $even$ $in$ $early$ $drafts$ because the fact that it is copied is easily forgotten as things move to more advanced drafts. “I forgot that these words came from somewhere else” is no excuse. Your supervisor will always check your thesis for plagiarism. You may consider doing this also yourself by using the tools provided from TU Delft. It is important to remember that plagiarism does not refer only to text but also to figures. This means, that you are not allowed to simply copy figures from published papers or books and paste them in your thesis. Even if you properly cite the source, still this is not allowed since already published figures are subject to copyrights. However, remaking a plot by e.g., extracting the data points from a published figure, is perfectly fine. The same with remaking a conceptual plot or design by adopting the general idea from a published figure but making significant changes.


\subsection{Make good use of feedback}
When you get feedback from your thesis supervisor (whether at the TU or at a company), please make good use of it. For example, when you see a suggested change, do not merely make the change, but rather consider whether or not you see the reason behind the change. If you see the reason behind the change, then you will hopefully incorporate this lesson into future writing to avoid more errors, rather than just changing that one error. Furthermore, it is worthwhile to think about whether this error is a systematic error, and, if it is, search through your document to fix it in other locations. As active supervisors, we sincerely hope that you are able to improve your writing, style, and grammar based on feedback that you receive on the the literature review. These improvements should be reflected in the writing, style, and grammar that you show in the first draft of your thesis. Likewise, we hope that providing feedback on early drafts will teach you more general lessons that can be applied on subsequent drafts, rather than just narrow changes that are specifically suggested. In this way, we hope that you will become a more accomplished professional after you have left the university and receive less feedback.

It will most likely help your thesis to get peer feedback. If nothing else, it will help you to identify the parts where you do not include sufficient background information. In the best case, your peers could help with language/grammar and with critical questions on content. If you have one or two friends who are doing their thesis at about the same time, then it would be efficient to make a pact that you read each other’s theses. Their effort is almost certainly a net time saver for the cost of having to read their work. Finally, there is a suggestion to divide the proof-reading effort. If you cannot find one person who is willing to read your whole thesis, then you may be able to find several friends who are willing to read one chapter each.


\section{Presentation style}
As you know from looking at the grading rubric, your presentation is an important part of the grade of your thesis. This is also a good way to practice a communication tool that will be extremely important for your career. You can work on your presentation in a similar way to your thesis: Figure out your conclusions first, then give enough information before that so that the audience can see how you reached to those conclusions and any background knowledge that is necessary to understand the developments. In the beginning of your presentation, you should give a clear motivation that should include a social and/or scientific reason for your interest in this subject. Ideally, you will be able to identify both a social AND a scientific reason.

The slides themselves should have as few words as possible and almost no equations. With every word and equation, you should ask yourself if it is needed. One way of developing a presentation is to start with a pen and a blank sheet of paper. Try to explain to yourself (or someone else) what you did, why it was important, how you did it, and what your conclusions are. Just explain, and sketch drawings, graphs, or figures as you need to in order to communicate your point. Look at the paper that you used at the end of the talk. These sketches should give you a pretty good idea of what should and should not be included into your slides. The primary information in the slides should be figures and pictures that are necessary to express your point. Remember, the figures you show need to tell your story. Avoid lists of numbers and tables as much as possible. In fact, presentations full of tables are not only boring but also extremely difficult to follow. All our tips about paying attention to proper notation, figures, and use of English apply also to your presentation. For example, make sure that all figures are legible and have proper units and notation. A typical rule of thumb is to make sure that all fonts are at least 14 point. Titles, axis labels, and numbers should also have legible font. All numbers should have units. 

It is important to practice. Even experienced presenters practice before giving a presentation. Be sure to time yourself. When/if you stutter or use filler words (um, uh…), consider what point you were struggling to bring across that caused you to use filler words and think about how that point could be improved. In general (not just for your thesis), one of the most important factors in giving a good presentation is knowing your audience. Even for the same content, you will give different presentations to world experts of your field than you would to a group of bachelor’s students. Think about what your audience will find interesting and what they will understand. In conferences, you can often surmise this information by looking into the titles in your conference session and figuring out what the other authors (and their respective audiences) know and finding interesting. For your thesis, you have two different audiences for the two main presentations. The audience of the green light meeting (note that not all departments have a green light meeting) is a small committee of scientific staff at a university, and their goal is to decide if you have made a sufficient contribution to be awarded a degree. Therefore, your presentation can assume that the audience has a lot of non-specialist knowledge and is interested in the scientific contributions. The audience at the defense could include your friends, peers, family, and general public. Therefore, you should make the presentation for a lower level of expertise/education and focus more on social impact and the broad lines of your research. Finally, when you prepare the presentation, focus on the main story. Say only what is needed to motivate and reach your conclusions. Do not try to describe everything you did, as there were certainly some dead ends or extra detail that the audience is not interested in.

There are a lot of resources (e.g., books, videos) available that can help you improve your presentation skills. We recommend you to watch the talk by Patrick Winston “How to speak” available from MIT OpenCourseWare\footnote{\url{https://www.youtube.com/watch?v=Unzc731iCUY}}.

\section{Day of the defense}
\subsection{Defense schedule/agenda}
The traditional defense in 3mE comprises:
\begin{itemize}
    \item Introduction by the graduation chairman (5 minutes)
    \item Presentation by the graduation student (20 minutes)
    \item Questions from the audience (10 minutes)
    \item Walk to the exam room (and take coffee!) (10 minutes)
    \item Closed exam session with the committee and the student (45 minutes)
    \item Exam evaluation by the committee (15 minutes)
    \item Feedback to the candidate (15 minutes)
    \item Diploma ceremony with audience (15 minutes)
\end{itemize}
The Ship and Offshore Structure (SAOS) section follows a different defense format which comprises:
\begin{itemize}
    \item Closed examination with the committee, including 15 minutes for the committee to decide on a grade (60 minutes)
    \item The student and committee are introduced to the audience (5 minutes)
    \item The student gives their public presentation before an audience (20 minutes)
    \item Questions from the audience (15 minutes)
    \item Committee finalizes grade, to include information from the presentation (5 minutes)
    \item Feedback is given to the student (10 minutes)
\end{itemize}
The traditional defense order within SAOS when $cum$ $laude$ is considered is:
\begin{itemize}
    \item Closed examination with the committee (60 minutes)
    \item Committee evaluates the student (30 minutes)
    \item The student and committee are introduced to the audience (5 minutes)
    \item The student gives their public presentation before an audience (20 minutes)
    \item Questions from the audience (15 minutes)
    \item Committee finalizes grade, to include information from the presentation (5 minutes)
    \item Feedback is given to the student (10 minutes)
\end{itemize}

\subsection{What to expect during the closed examination}
Usually, the examination committee consists of 2-4 examiners including your supervisor. Thus, you should expect that at least 1 examiner is not an expert in your field. Although the depth and variety of questions the examiners will ask you varies a lot (based, e.g., on the composition of the committee, the subject), some questions are more common than others. For example, you may be asked "What is the origin of Equation X?”. It is often not necessary to be able to derive everything on the spot, but it is important that you are able to understand the basic principles and assumptions/approximations behind the equation/theory (e.g., “this Equation follows from an energy balance”, or “Newton’s second law”) and also that you know the underlying assumptions (e.g., “we assume here that the liquid is incompressible”, “the ideal gas law applies”). Another frequently asked question is about the validity of the assumptions used in a model or a theory or for interpreting experimental results. You can be asked about the range at which these assumptions are valid or which of them are the most important. Examiners may also ask you to interpret/explain what you show in a figure (e.g., explain why $X$ is going down while $Y$ up), or what is the uncertainty in a quantity and why you have this uncertainty. Such questions are easy to anticipate and prepare for. Essentially, what you write in your thesis is the basis for questions by the examiners, therefore, you can expect a question on everything that you write.

Always give clear answers to the examiners without providing more information than needed. Don’t rush and answer; it is far better to pause for a few seconds to consider the answer than give an inappropriate or incorrect answer. If you do not know the answer to a question, then say it and maybe ask the examiner for a hint. Do not try to invent an answer and make up a story to hide the fact that you do not know the answer. This will neither give a positive impression to the examiner nor help you find the actual answer.  


\section{A few final remarks}
Always remember that you should be the master of your own work/project. Prepare well for the meetings and clearly formulate your questions and doubts. Be open to your supervisor about things you are struggling with. He/she is there to help and guide you. During your meetings and when receiving feedback, always remember that your supervisor is working with you to get the best thesis possible. The thesis supervisor is not your adversary, even if they can sometimes ask for difficult things. Always try to write/formulate things in a clear and succinct way. If you feel unhappy with something or experience unwanted/abusive behavior, always act.

Finally, common sense is a valuable guide. Try to use it as much as possible during your graduation project too. For example, it is common sense that you should not plagiarize in your thesis. It is also common sense that you should be on time for your meetings.

\section{Acknowledgements}
For this document, we would like to acknowledge the helpful advice of Henk Polinder, Thijs Vlugt, Marion Tissier, and the many students who have read through and given feedback.


\bibliographystyle{plain}
% \stepcounter{section} % Increase counter (section) by one step
% \addcontentsline{toc}{section}{\thesection \quad References} % Add to ToC (at the section level)
\bibliography{bibliography}

%\end{multicols}
\end{document}